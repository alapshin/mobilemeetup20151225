\documentclass{beamer}

\usepackage{fontspec}
\setmainfont{Liberation Serif}
\setsansfont{Liberation Sans}
\setmonofont{Liberation Mono}
\newfontfamily{\cyrillicfont}{Liberation Serif}
\newfontfamily{\cyrillicfonttt}{Liberation Mono}

\usepackage{polyglossia}
\setmainlanguage{russian}
\setotherlanguage{english}

\usepackage{graphicx}

\usepackage{hyperref}
\hypersetup{unicode=true,pdftitle={Kotlin}}

\usepackage{tikz}
\usetikzlibrary{positioning,timeline}
\usepackage{pgf-pie}
\usepackage{ulem}
\usepackage{xcolor}
\usepackage{easylist}


\author{Андрей Лапшин}
\date{2015-12-25}
\title{Новый язык программирования от JetBrains}
\titlegraphic{\includegraphics{pics/logo_kotlin}}

% pgf-pie setup
\newif\ifpienumberinlegend
\pgfkeys{/number in legend/.code=
    \expandafter\let\expandafter\ifpienumberinlegend
    \csname if#1\endcsname
    \ifpienumberinlegend
    \let\legendbeforenumber\beforenumber
    \let\legendafternumber\afternumber
    \def\beforenumber##1\afternumber{}%
    \fi,
    /number in legend/.default=true
}

\makeatletter
\definecolor{kotlin}{HTML}{4D6BAC}
\setbeamercolor{titlelike}{fg=kotlin}
\setbeamercolor{itemize item}{fg=kotlin}
\setbeamertemplate{navigation symbols}{}
\setbeamertemplate{title page}
{
  \vbox{}
  \vfill
  \begin{centering}
    \begin{beamercolorbox}[sep=8pt,center]{title}
      {\usebeamercolor[fg]{titlegraphic}\inserttitlegraphic\par}
      \vskip0.25em%
      \usebeamerfont{title}\inserttitle\par%
      \ifx\insertsubtitle\@empty%
      \else%
        \vskip0.25em%
        {\usebeamerfont{subtitle}\usebeamercolor[fg]{subtitle}\insertsubtitle\par}%
      \fi%
    \end{beamercolorbox}%
    \vskip1em\par
    \begin{beamercolorbox}[sep=8pt,center]{author}
      \usebeamerfont{author}\insertauthor
    \end{beamercolorbox}
    \begin{beamercolorbox}[sep=8pt,center]{institute}
      \usebeamerfont{institute}\insertinstitute
    \end{beamercolorbox}
    \begin{beamercolorbox}[sep=8pt,center]{date}
      \usebeamerfont{date}\insertdate
    \end{beamercolorbox}\vskip0.5em
  \end{centering}
  \vfill
}
\makeatother

\begin{document}

\begin{frame}
    \titlepage
\end{frame}

\begin{frame}
    \frametitle{Что это?}
    Cтатический типизированный язык программирования для JVM, Android
    и Javascript от JetBrains:
    \begin{itemize}
        \item AppCode
        \item IntelliJ IDEA
            \begin{itemize}
                \item Android Studio
            \end{itemize}
    \end{itemize}
\end{frame}

\begin{frame}
    \frametitle{Зачем нужен новый язык?}
    На Android на данный момент доступна Java 6 (Java 7, если minSdkVersion=19),
    в то время как уже выпущена Java 8 и разрабатывается Java 9 (~сентябрь 2016)

    \begin{figure}
        \begin{tikzpicture}[timespan={},scale=0.45,font=\tiny]
            \timeline[custom interval=true]{2006,\hspace{2em},2008,\hspace{3em},2011,\hspace{2em},2013,2014,\hspace{4em}, 20xx}
            % put here the phases
            \begin{phases}
                \tikzset{phase appearance/.append style={
                        circle,
                        opacity=0.0,
                        minimum size=\involvdegree}
                }
                \phase{between week=1 and 1 in 0.0,involvement degree=0.0,phase appearance={opacity=1.0}}
                \phase{between week=3 and 3 in 0.5,involvement degree=0.0}
                \phase{between week=5 and 5 in 0.5,involvement degree=0.0}
                \phase{between week=7 and 7 in 0.5,involvement degree=0.0}
                \phase{between week=8 and 8 in 0.5,involvement degree=0.0}
                \phase{between week=10 and 10 in 1.0,involvement degree=0.0}
            \end{phases}

            % put here the milestones
            \addmilestone{at=phase-1.90,direction=90:1cm,circle radius=0.0,text={Java 6},text options={above}}
            \addmilestone{at=phase-2.270,direction=270:1cm,text={Android 1.0},text options={below}}
            \addmilestone{at=phase-3.90,direction=90:1cm,text={Java 7},text options={above}}
            \addmilestone{at=phase-4.270,direction=270:1cm,text={Android 4.4},text options={below}}
            \addmilestone{at=phase-5.90,direction=90:1cm,text={Java 8},text options={above}}
            \addmilestone{at=phase-6.270,direction=270:1cm,text={Android ???},text options={below}}
        \end{tikzpicture}
    \end{figure}

    \begin{figure}
        \begin{tikzpicture}[scale=0.45]
            \pie[text=legend, before number=\ (, after number=\,\%), number in legend]{
                0.5/Marshmallow,
                29.5/Lollipop,
                36.6/Kitkat,
                26.9/Jelly Bean,
                2.9/Ice Cream Sandwich,
                3.6/Gingerbread + Froyo
            }
        \end{tikzpicture}
    \end{figure}
\end{frame}

\begin{frame}
    \frametitle{Зачем нужен новый язык?}
    Недостатки Java 6
    \begin{itemize}
        \item \alt<-1>{Нет javax.time}{\sout{Нет javax.time} \textcolor{kotlin}{ThreeTenABP}}
        \item \alt<-2>{Нет Streams API}{\sout{Нет Streams API} \textcolor{kotlin}{StreamSupport/RxJava}}
        \item \alt<-3>{Нет лямбд}{\sout{Нет лямбд} \textcolor{kotlin}{Retrolambda}}
        \item \only<-4>{Нет try-with-resources}\only<5>{\sout{Нет try-with-resources} \textcolor{kotlin}{Retrolambda/minSdkVersion=19}}
    \end{itemize}
\end{frame}

\begin{frame}
    \frametitle{Зачем нужен новый язык?}
    Недостатки Java в целом
    \begin{itemize}
        \item Добавление новой функциональности к существующим классам:
            наследование или куча Util-классов
        \item Проблемы при работе null
        \item Изменяемые структуры данных по-умолчанию
        \item Многословие
    \end{itemize}
\end{frame}

\begin{frame}
    \frametitle{Почему именно Kotlin}
    \begin{itemize}
        \item 100\% совместим с Java
        \item Отличные инструменты разработки
        \item Сравнительно небольшой размер рантайма (\approx 950K) и количество методов (\approx 7900)
    \end{itemize}
\end{frame}

\begin{frame}
\end{frame}

\end{document}
