\documentclass{beamer}

\setbeameroption{show notes}
\makeatletter
\definecolor{kotlin}{HTML}{4D6BAC}
\setbeamercolor{titlelike}{fg=kotlin}
\setbeamercolor{itemize item}{fg=kotlin}
\setbeamercolor{note page}{bg=white}
\setbeamercolor{note title}{bg=white}
\setbeamertemplate{navigation symbols}{}
\setbeamertemplate{title page}
{
  \vbox{}
  \vfill
  \begin{centering}
    \begin{beamercolorbox}[sep=8pt,center]{title}
      {\usebeamercolor[fg]{titlegraphic}\inserttitlegraphic\par}
      \vskip0.25em%
      \usebeamerfont{title}\inserttitle\par%
      \ifx\insertsubtitle\@empty%
      \else%
        \vskip0.25em%
        {\usebeamerfont{subtitle}\usebeamercolor[fg]{subtitle}\insertsubtitle\par}%
      \fi%
    \end{beamercolorbox}%
    \vskip1em\par
    \begin{beamercolorbox}[sep=8pt,center]{author}
      \usebeamerfont{author}\insertauthor
    \end{beamercolorbox}
    \begin{beamercolorbox}[sep=8pt,center]{institute}
      \usebeamerfont{institute}\insertinstitute
    \end{beamercolorbox}
    \begin{beamercolorbox}[sep=8pt,center]{date}
      \usebeamerfont{date}\insertdate
    \end{beamercolorbox}\vskip0.5em
  \end{centering}
  \vfill
}
\makeatother

\usepackage{fontspec}
\setmainfont{Liberation Serif}
\setsansfont{Liberation Sans}
\setmonofont{Liberation Mono}
\newfontfamily{\cyrillicfont}{Liberation Serif}
\newfontfamily{\cyrillicfonttt}{Liberation Mono}

% \usepackage{unicode-math}
% \setmathfont{STIX Math}

\usepackage{polyglossia}
\setmainlanguage{russian}
\setotherlanguage{english}

\usepackage{graphicx}

\usepackage{hyperref}
\hypersetup{unicode=true, colorlinks=true}

\usepackage{tikz}
\usetikzlibrary{positioning,timeline}
\usepackage{pgf-pie}
% pgf-pie setup
\newif\ifpienumberinlegend
\pgfkeys{/number in legend/.code=
    \expandafter\let\expandafter\ifpienumberinlegend
    \csname if#1\endcsname
    \ifpienumberinlegend
    \let\legendbeforenumber\beforenumber
    \let\legendafternumber\afternumber
    \def\beforenumber##1\afternumber{}%
    \fi,
    /number in legend/.default=true
}

\usepackage{ulem}
\usepackage{xcolor}
\usepackage{minted}


\author{Андрей Лапшин}
\institute{KSI}
\date{2015-12-25}
\title{Новый язык программирования от JetBrains}
\titlegraphic{\includegraphics{pics/logo_kotlin}}

\begin{document}

\begin{frame}
    \titlepage
    \note{Привет. Меня зовут Андрей Лапшин, я работаю C++/Android программистом
        в компании KSI и сегодня я вам попытаюсь рассказать про новый язык
        програмирования Kotlin.}
\end{frame}

\begin{frame}
    \frametitle{Что это?}
    Cтатический типизированный язык программирования для JVM, Android
    и Javascript от JetBrains:
    \begin{itemize}
        \item AppCode
        \item IntelliJ IDEA
            \begin{itemize}
                \item Android Studio
            \end{itemize}
    \end{itemize}
    \note{Kotlin это статически типизированный язык программирования для JVM,
        Android и браузера от компании JetBrains. Компания JetBrains большинству
        из вас известна как разработчик ряда IDE таких как InteliiJ IDEA, AppCode,
        PhpStorm и т. п. Android Studio построена на базе открытой версии
        InteliiJ IDEA. Кроме IDE компания также выпускает ряд серверных
        приложений типа YouTrack, TeamCity. Большая часть продуктов JetBrains
        написана на Java, но в 2010 году существующих на тот момент фич Java
        видимо стало не хватать, поэтому они решили начать разрабатывать новый язык.}
\end{frame}

\begin{frame}
    \frametitle{Зачем нужен новый язык?}
    На Android на данный момент доступна Java 6 (Java 7, если minSdkVersion=19),
    в то время как уже выпущена Java 8 и разрабатывается Java 9 (~сентябрь 2016)

    \begin{figure}
        \begin{tikzpicture}[timespan={},scale=0.45,font=\tiny]
            \timeline[custom interval=true]{2006,\hspace{2em},2008,\hspace{3em},2011,\hspace{2em},2013,2014,\hspace{4em}, 20xx}
            % put here the phases
            \begin{phases}
                \tikzset{phase appearance/.append style={
                        circle,
                        opacity=0.0,
                        minimum size=\involvdegree}
                }
                \phase{between week=1 and 1 in 0.0,involvement degree=0.0,phase appearance={opacity=1.0}}
                \phase{between week=3 and 3 in 0.5,involvement degree=0.0}
                \phase{between week=5 and 5 in 0.5,involvement degree=0.0}
                \phase{between week=7 and 7 in 0.5,involvement degree=0.0}
                \phase{between week=8 and 8 in 0.5,involvement degree=0.0}
                \phase{between week=10 and 10 in 1.0,involvement degree=0.0}
            \end{phases}

            % put here the milestones
            \addmilestone{at=phase-1.90,direction=90:1cm,circle radius=0.0,text={Java 6},text options={above}}
            \addmilestone{at=phase-2.270,direction=270:1cm,text={Android 1.0},text options={below}}
            \addmilestone{at=phase-3.90,direction=90:1cm,text={Java 7},text options={above}}
            \addmilestone{at=phase-4.270,direction=270:1cm,text={Android 4.4},text options={below}}
            \addmilestone{at=phase-5.90,direction=90:1cm,text={Java 8},text options={above}}
            \addmilestone{at=phase-6.270,direction=270:1cm,text={Android ???},text options={below}}
        \end{tikzpicture}
    \end{figure}

    \begin{figure}
        \begin{tikzpicture}[scale=0.45]
            \pie[text=legend, before number=\ (, after number=\,\%), number in legend]{
                0.5/Marshmallow,
                29.5/Lollipop,
                36.6/Kitkat,
                26.9/Jelly Bean,
                2.9/Ice Cream Sandwich,
                3.6/Gingerbread + Froyo
            }
        \end{tikzpicture}
    \end{figure}
    \note{У любого здравомыслящего человека после того как, он слышит о новом
        языке программирования сразу возникает вопрос: "Зачем и что в 2015 году
        не устраивает в Java?" Однако при разработке под Android мы по-прежнему
        вынуждены работать с Java как в 2010 году. В то время как на десктопе и
        сервере разработчикам доступна Java 8, на Android мы можем использовать
        в лучшем случае Java 7, при условии что мы забудем про ~25\% пользователей
        на Jelly Bean.}
\end{frame}

\begin{frame}
    \frametitle{Зачем нужен новый язык?}
    Недостатки Java 6
    \begin{itemize}
        \item \alt<-1>{Нет javax.time}{\sout{Нет javax.time} \textcolor{kotlin}{ThreeTenABP}}
        \item \alt<-2>{Нет Streams API}{\sout{Нет Streams API} \textcolor{kotlin}{StreamSupport/RxJava}}
        \item \alt<-3>{Нет лямбд}{\sout{Нет лямбд} \textcolor{kotlin}{Retrolambda}}
        \item \only<-4>{Нет try-with-resources}\only<5>{\sout{Нет try-with-resources} \textcolor{kotlin}{Retrolambda/minSdkVersion=19}}
    \end{itemize}
    \note{В результате на Android недоступны например, следуюшие API. На самом
        вопрос нехватки каких-то библиотек решаем. Например, вместо javax.time
        можно использовать бэкпорт этой библиотеки. Для Streams API также доступен
        бэкпорт, либо можно использовать библиотеку RxJava которая по целям
        совпадает со Stream API но умеет многое другое. Даже некоторые
        новые фичи языка становятся доступными, если использовать дополнительные
        инструменты.}
\end{frame}

\begin{frame}
    \frametitle{Зачем нужен новый язык?}
    Недостатки Java в целом
    \note<1>{
        Однако даже с учетом этого у Java как языка в целом есть
        недостатки, которые в сочетании с недостатками Android как платформы
        затрудняют разработку.
    }
    \begin{itemize}
        \item<1-> Необходимость наследования от системных классов
            \note<1>{
                Первая проблема особенно заметна на Android, достаточно взглянуть на
                иерархию Activity-классов (Activity -> FragmentedActivity -> AppCompatActivity)
            }
        \item<2-> Невозможность расширения функциональности существующих классов
            (накапливается множество мелких Util-классов)
            \note<2>{
                Вторая проблема актуальна не только для Android. Например, вам нужно добавить какой-то
                вспомогательный метод для класса (валидация) но к классу вы его добавить не можете, так
                как класс лежит в какой-то библиотеке. Нужно либо наследоваться от класса,
                либо создать Util-класс в котором определен статический метод с нужным функционалом.
                Упомянуть TextUtil. В результате накапливается множество мелких вспомогательных классов.
            }
        \item<3-> Проблемы при работе c null
            \note<3>{
                Третья проблема это наличие языке null (так называемая "ошибка на миллиард долларов").
                Если у вас есть какой-то метод возращаюший объект, вы заранее не знаете, вернет ли он
                объект или null. Нужно либо смотреть документацию, либо явно проверять возвращаемое значение.
                В Android с этим особенно сложно, так как для оптимизации производительности,
                куча системных методов может принимать и возвращать null.
            }
        \item<4-> Многословие
            \note<4>{
                Последний недостаток Java - это многословие. Порой для простейших вещей
                нужно написать десяток строк
            }
    \end{itemize}
\end{frame}

\begin{frame}
    \frametitle{Почему именно Kotlin}
    \begin{itemize}
        \item 100\% совместим с Java
        \item Отличные инструменты разработки
        \item Сравнительно небольшой размер рантайма (\approx 950K) и количество методов (\approx 7900)
    \end{itemize}
    \note{
        Предположим, что вы поработали с джавой пару лет и вам хочется чего-то
        более современного. Почему именно Kotlin, а например не Scala или Groovy
        или любой из других JVM-языков?
    }
\end{frame}

\begin{frame}[fragile, t]
    \frametitle{Базовый синтакс}
    \onslide<1->
    Функции
    \begin{minted}[gobble=8]{kotlin}
        fun sum(a: Int, b: Int): Int {
            return a + b
        }
    \end{minted}

    \onslide<2->
    \begin{minted}[gobble=8]{kotlin}
        fun diff(a: Int, b: Int) = a - b
    \end{minted}

    \onslide<3->
    Переменные
    \begin{minted}[gobble=8]{kotlin}
        val a: Int = 1
        val b = 1 // Автовыведение типа

        var x = 5 // Изменяемая переменная
        x += 1
    \end{minted}
    \note{Обсудили зачем новый язык, теперь давайте посмотрим на синтаксиса
        примеры кода
        \\
        Функции в Kotlin объвляются с помощью ключевого слова fun, порядок имени
        аргументов и их типов отличается Java и напоминает Pascal, сначала имя
        аргумента, затем тип. В отличии от Java Kotiln допускает функции вне классов
        Указать на возможность пропуска точки с запятой
        \\
        Переменные в Kotlin могут быть объявлены с помощью ключевых слов val и
        var. val - неизменяемая переменная, var - изменяемая. И поддерживают выведение типа
    }
\end{frame}

\begin{frame}[fragile, t]
    \frametitle{Классы, свойства}
    \begin{overprint}
        \onslide<1>
        \begin{minted}[gobble=12]{kotlin}
            class Person {
                val name = "Vasya Pupkin"
            }
        \end{minted}

        \onslide<2>
        \begin{minted}[gobble=12]{kotlin}
            class Person constructor(name: String) {
                val name = name
            }
        \end{minted}

        \onslide<3>
        \begin{minted}[gobble=12]{kotlin}
            class Person constructor(val name: String) {
            }
        \end{minted}

        \onslide<4>
        \begin{minted}[gobble=12]{kotlin}
            class Person(val name: String) {
            }
            val person = Person("Joe Smith")
        \end{minted}
    \end{overprint}
    \note{
        Классы в Kotlin объявляются с помощью ключевого слова class.
        Классы могут иметь свойства, которые объявляются с помощью val или var.
        val - неизменяемое свойство, var - изменяемое. Для свойств автоматически
        генерируются геттеры и сеттеры. Классы могут иметь конструкторы,
        аргументы конструктора могут использоваться при инициалиации свойств класса
    }
\end{frame}

\begin{frame}[fragile, t]
    \frametitle{Работа с null}
    \onslide<1->
    \begin{minted}[gobble=8]{kotlin}
        var a: String = "abc"
        a = null // ошибка компиляции
    \end{minted}
    \onslide<2->
    \begin{minted}[gobble=8]{kotlin}
        var b: String? = "abc"
        b = null // ok
    \end{minted}
    \onslide<3->
    \begin{minted}[gobble=8]{kotlin}
        val l1 = a.length // ok
        val l2 = b.length // ошибка: 'b' может быть null
    \end{minted}
    \note{
        В Kotlin тип переменных, которые могут содержать null, отличается от
        типа переменных которые не могут. При обращении к свойствам или методам
        объекта, который может быть null компилятор также выдает ошибку
    }
\end{frame}

\begin{frame}[fragile, t]
    \frametitle{Работа с null}
    \note{
        Есть три способа получения значения, которое лежит в переменной
        способной хранить null.
    }
    \onslide<1->
    Явная проверка на null
    \note{
        Первый вариант проверка на null в условии. Компилятор учитывает эту
        проверку и позволяет обращаться с к свойствам и методам соответствующего
        объекта в соответсвующей ветке
    }
    \begin{minted}[gobble=8]{kotlin}
        val l = if (b != null) b.length else -1
    \end{minted}

    \onslide<2->
    Использование оператора ?
    \note{
        Второй вариант - использование оператора ?, в том случае если в переменной
        лежит null обращение к свойствам объекта также вернет null, если в переменной
        лежит значение то отработает как обычно
    }
    \begin{minted}[gobble=8]{kotlin}
        val l = b?.length
        val name = bob?.department?.head?.name
    \end{minted}

    \onslide<3->
    Использование оператора !!
    \note{
        Третий вариант - использование оператора !! В этом случае при обращении
        к свойствам переменной в которой лежит null будет брошено NPE.
    }
    \begin{minted}[gobble=8]{kotlin}
        val l = b!!.length()
    \end{minted}
\end{frame}

\begin{frame}[fragile, t]
    \frametitle{Extension functions}
    \onslide<1->
    \begin{minted}[gobble=8]{java}
        static boolean isVeryLong(String s) {
            return s.length() > 80;
        }
        boolean tooLong = TextUtils.isTooLong(s);
    \end{minted}

    \onslide<2->
    \begin{minted}[gobble=8]{kotlin}
        fun String.isTooLong() {
            return length() > 80
        }
        val tooLong = s.isTooLong()
    \end{minted}
    \note{
        В Java это достигается за счет создания вспомогательного класса со
        статическим методом. В Kotlin мы можем объявить функцию-расширение на
        нужном типе и к ней можно будет обращаться как к методу класса.
    }
\end{frame}

\begin{frame}[fragile, t]
    \frametitle{Лямбды}
    \onslide<1->
    \begin{minted}[gobble=8, fontsize=\small]{java}
        button.setOnClickListener(new View.OnClickListener() {
            public void onCLick(View v) {
                Toast.makeText(MyActivity.this, "Some text",
                Toast.LENGTH_SHORT).show();
            }
        });
    \end{minted}

    \onslide<2->
    \begin{minted}[gobble=8, fontsize=\small]{java}
        button.setOnClickListener(v ->
            Toast.makeText(MyActivity.this, "Some text",
            Toast.LENGTH_SHORT).show();
        });
    \end{minted}

    \onslide<3->
    \begin{minted}[gobble=8, fontsize=\small]{kotlin}
        button.setOnClickListener({ v: View ->
            Toast.makeText(MyActivity.this, "Some text",
            Toast.LENGTH_SHORT).show()
        })
    \end{minted}

    \onslide<4->
    \begin{minted}[gobble=8, fontsize=\small]{kotlin}
        button.setOnClickListener {
            Toast.makeText(MyActivity.this, "Some text",
            Toast.LENGTH_SHORT).show()
        }
    \end{minted}
    \note{
        Лямбды отлично помогают в написании более краткого кода и особенно
        полезны на Android где множество методов принимают в качестве параметра
        колбэки. Kotlin лямбды поддерживает из коробки.
    }
\end{frame}


\begin{frame}[fragile, t]
    \frametitle{Функции высшего порядка}
    \note{
        Функции высшего порядка - это функции которые принимают в качестве
        аргументов или возвращают другие функции. Рассмотрим пример определения
        функии высшего порядка, которая позволит фильтровать список по заданному
        критерию.
    }
    \begin{overprint}
        \onslide<1>
        \begin{minted}[gobble=12]{kotlin}
            fun<T> List<T>.filter(predicate: (T) -> Boolean): List<T> {
                // ...
            }
        \end{minted}

        \onslide<2>
        \begin{minted}[gobble=12]{kotlin}
            fun<T> List<T>.filter(predicate: (T) -> Boolean): List<T> {
                val resultList = ArrayList<T>()
                for (item in this) {
                    if (predicate(item)) {
                        resultList.add(item)
                    }
                }

                return resultList
            }
        \end{minted}

        \onslide<3>
        \begin{minted}[gobble=12]{kotlin}
            fun<T> List<T>.filter(predicate: (T) -> Boolean): List<T> {
                val resultList = ArrayList<T>()
                for (item in this) {
                    if (predicate(item)) {
                        resultList.add(item)
                    }
                }

                return resultList
            }

            val numbers = listOf(1, 2, 3, 4, 5, 6, 7, 8, 9, 0)
            val oddNumbers = numbers.filter { it % 2 != 0 }
        \end{minted}
    \end{overprint}
\end{frame}

\begin{frame}[fragile, t]
    \frametitle{Функции высшего порядка}
    \note<1>{
        Напоследок пример кода, который объединяет в себе лямбды,
        функции-расширения и функции высшего порядка.
    }
    \begin{overprint}
        \onslide<1>
        \note<1>{
            Предположим нам нужно удалить запись из БД, для этого напишем
            такой код
        }
        \begin{minted}[gobble=12, fontsize=\small]{java}
            db.beginTransaction();
            try {
                db.delete("users", "first_name = ?", new String[] { "Vasya" });
            } finally {
                db.endTransaction();
            }
        \end{minted}

        \onslide<2>
        \note<2>{
            В этом коде ошибка, забыли указать БД на успешное завершение
            транзакции
        }
        \begin{minted}[gobble=12, fontsize=\small]{java}
            db.beginTransaction();
            try {
                db.delete("users", "first_name = ?", new String[] { "Vasya" });
                db.setTransactionSuccessful();
            } finally {
                db.endTransaction();
            }
        \end{minted}

        \onslide<3>
        \note<3>{
            Посмотрим как можно упростить этот код, используя Kotlin.
            Определим функцию-расширение inTransaction на классе SQLiteDatabase,
            которая в качестве аргумента принимают функцию без аргументов и возвращаемого значения
        }
        \begin{minted}[gobble=12, fontsize=\small]{kotlin}
            fun SQLiteDatabase.inTransaction(func: () -> Unit) {
                beginTransaction()
                try {
                    func()
                    setTransactionSuccessful()
                } finally {
                    endTransaction()
                }
            }

            db.inTransaction {
                db.delete("users", "first_name = ?", arrayOf("Vasya")
            }
        \end{minted}

        \onslide<4>
        \note<4>{
            Повторное указание db внутри inTransaction кажется избыточным.
            Может от него можно избавиться? Переопределим немного
            функцию-расширение. Теперь она в качестве аргумента принимают функцию,
            которая в свою очередь принимает SQLiteDatabase и возвращает Unit
        }
        \begin{minted}[gobble=12, fontsize=\small]{kotlin}
            fun SQLiteDatabase.inTransaction(
                        func: (SQLiteDatabase) -> Unit) {
                beginTransaction()
                try {
                    func(this)
                    setTransactionSuccessful()
                } finally {
                    endTransaction()
                }
            }

            db.inTransaction {
                it.delete("users", "first_name = ?", arrayOf("Vasya")
            }
        \end{minted}

        \onslide<5>
        \note<5>{
            Упростим ещё немного. Теперь функция inTransaction принимает
            в качестве аргумента другую функцию-раширение на классе SQLiteDatabase
        }
        \begin{minted}[gobble=12, fontsize=\small]{kotlin}
            fun SQLiteDatabase.inTransaction(
                    func: SQLiteDatabase.() -> Unit) {
                beginTransaction()
                try {
                    this.func()
                    setTransactionSuccessful()
                } finally {
                    endTransaction()
                }
            }

            db.inTransaction {
                delete("users", "first_name = ?", arrayOf("Vasya")
            }
        \end{minted}

        \onslide<6>
        \note<6>{
            Для повышения производительности объявим эту функцию inline, чтобы
            kotlin не генерировал для каждого вызова отдельный класс
        }
        \begin{minted}[gobble=12, fontsize=\small]{kotlin}
            inline fun SQLiteDatabase.inTransaction(
                    func: SQLiteDatabase.() -> Unit) {
                beginTransaction()
                try {
                    this.func()
                    setTransactionSuccessful()
                } finally {
                    endTransaction()
                }
            }

            db.inTransaction {
                delete("users", "first_name = ?", arrayOf("Vasya")
            }
        \end{minted}
    \end{overprint}

\end{frame}

\begin{frame}
    \frametitle{Документация и полезные ссылки}
    \begin{description}
        \item Сайт и документация: \\
            \url{https://kotlinlang.org}
        \item Онлайн IDE: \\
            \url{http://try.kotlinlang.org}
        \item Презентация Джейка Уортона (Jake Wharton) о языке (англ.): \\
            \href{http://www.youtube.com/watch?v=A2LukgT2mKc}{Android Development with Kotlin}
        \item Мотивации для использования и сравнение с другими языками под JVM (англ.): \\
            \href{https://docs.google.com/document/d/1ReS3ep-hjxWA8kZi0YqDbEhCqTt29hG8P44aA9W0DM8/edit?hl=en&forcehl=1}{Using Project Kotlin for Android}
    \end{description}
\end{frame}

\end{document}
