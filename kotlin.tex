\documentclass{beamer}

\usepackage{fontspec}
\setmainfont{Liberation Serif}
\setsansfont{Liberation Sans}
\setmonofont{Liberation Mono}
\newfontfamily{\cyrillicfont}{Liberation Serif}
\newfontfamily{\cyrillicfonttt}{Liberation Mono}

\usepackage{unicode-math}
\setmathfont{STIX Math}

\usepackage{polyglossia}
\setmainlanguage{russian}
\setotherlanguage{english}

\usepackage{graphicx}

\usepackage{hyperref}
\hypersetup{unicode=true, colorlinks=true}

\usepackage{tikz}
\usetikzlibrary{positioning,timeline}
\usepackage{pgf-pie}
\usepackage{ulem}
\usepackage{xcolor}
\usepackage{minted}


\author{Андрей Лапшин}
\date{2015-12-25}
\title{Новый язык программирования от JetBrains}
\titlegraphic{\includegraphics{pics/logo_kotlin}}

% pgf-pie setup
\newif\ifpienumberinlegend
\pgfkeys{/number in legend/.code=
    \expandafter\let\expandafter\ifpienumberinlegend
    \csname if#1\endcsname
    \ifpienumberinlegend
    \let\legendbeforenumber\beforenumber
    \let\legendafternumber\afternumber
    \def\beforenumber##1\afternumber{}%
    \fi,
    /number in legend/.default=true
}

\makeatletter
\definecolor{kotlin}{HTML}{4D6BAC}
\setbeamercolor{titlelike}{fg=kotlin}
\setbeamercolor{itemize item}{fg=kotlin}
\setbeamertemplate{navigation symbols}{}
\setbeamertemplate{title page}
{
  \vbox{}
  \vfill
  \begin{centering}
    \begin{beamercolorbox}[sep=8pt,center]{title}
      {\usebeamercolor[fg]{titlegraphic}\inserttitlegraphic\par}
      \vskip0.25em%
      \usebeamerfont{title}\inserttitle\par%
      \ifx\insertsubtitle\@empty%
      \else%
        \vskip0.25em%
        {\usebeamerfont{subtitle}\usebeamercolor[fg]{subtitle}\insertsubtitle\par}%
      \fi%
    \end{beamercolorbox}%
    \vskip1em\par
    \begin{beamercolorbox}[sep=8pt,center]{author}
      \usebeamerfont{author}\insertauthor
    \end{beamercolorbox}
    \begin{beamercolorbox}[sep=8pt,center]{institute}
      \usebeamerfont{institute}\insertinstitute
    \end{beamercolorbox}
    \begin{beamercolorbox}[sep=8pt,center]{date}
      \usebeamerfont{date}\insertdate
    \end{beamercolorbox}\vskip0.5em
  \end{centering}
  \vfill
}
\makeatother

\begin{document}

\begin{frame}
    \titlepage
\end{frame}

\begin{frame}
    \frametitle{Что это?}
    Cтатический типизированный язык программирования для JVM, Android
    и Javascript от JetBrains:
    \begin{itemize}
        \item AppCode
        \item IntelliJ IDEA
            \begin{itemize}
                \item Android Studio
            \end{itemize}
    \end{itemize}
\end{frame}

\begin{frame}
    \frametitle{Зачем нужен новый язык?}
    На Android на данный момент доступна Java 6 (Java 7, если minSdkVersion=19),
    в то время как уже выпущена Java 8 и разрабатывается Java 9 (~сентябрь 2016)

    \begin{figure}
        \begin{tikzpicture}[timespan={},scale=0.45,font=\tiny]
            \timeline[custom interval=true]{2006,\hspace{2em},2008,\hspace{3em},2011,\hspace{2em},2013,2014,\hspace{4em}, 20xx}
            % put here the phases
            \begin{phases}
                \tikzset{phase appearance/.append style={
                        circle,
                        opacity=0.0,
                        minimum size=\involvdegree}
                }
                \phase{between week=1 and 1 in 0.0,involvement degree=0.0,phase appearance={opacity=1.0}}
                \phase{between week=3 and 3 in 0.5,involvement degree=0.0}
                \phase{between week=5 and 5 in 0.5,involvement degree=0.0}
                \phase{between week=7 and 7 in 0.5,involvement degree=0.0}
                \phase{between week=8 and 8 in 0.5,involvement degree=0.0}
                \phase{between week=10 and 10 in 1.0,involvement degree=0.0}
            \end{phases}

            % put here the milestones
            \addmilestone{at=phase-1.90,direction=90:1cm,circle radius=0.0,text={Java 6},text options={above}}
            \addmilestone{at=phase-2.270,direction=270:1cm,text={Android 1.0},text options={below}}
            \addmilestone{at=phase-3.90,direction=90:1cm,text={Java 7},text options={above}}
            \addmilestone{at=phase-4.270,direction=270:1cm,text={Android 4.4},text options={below}}
            \addmilestone{at=phase-5.90,direction=90:1cm,text={Java 8},text options={above}}
            \addmilestone{at=phase-6.270,direction=270:1cm,text={Android ???},text options={below}}
        \end{tikzpicture}
    \end{figure}

    \begin{figure}
        \begin{tikzpicture}[scale=0.45]
            \pie[text=legend, before number=\ (, after number=\,\%), number in legend]{
                0.5/Marshmallow,
                29.5/Lollipop,
                36.6/Kitkat,
                26.9/Jelly Bean,
                2.9/Ice Cream Sandwich,
                3.6/Gingerbread + Froyo
            }
        \end{tikzpicture}
    \end{figure}
\end{frame}

\begin{frame}
    \frametitle{Зачем нужен новый язык?}
    Недостатки Java 6
    \begin{itemize}
        \item \alt<-1>{Нет javax.time}{\sout{Нет javax.time} \textcolor{kotlin}{ThreeTenABP}}
        \item \alt<-2>{Нет Streams API}{\sout{Нет Streams API} \textcolor{kotlin}{StreamSupport/RxJava}}
        \item \alt<-3>{Нет лямбд}{\sout{Нет лямбд} \textcolor{kotlin}{Retrolambda}}
        \item \only<-4>{Нет try-with-resources}\only<5>{\sout{Нет try-with-resources} \textcolor{kotlin}{Retrolambda/minSdkVersion=19}}
    \end{itemize}
\end{frame}

\begin{frame}
    \frametitle{Зачем нужен новый язык?}
    Недостатки Java в целом
    \begin{itemize}
        \item Добавление новой функциональности к существующим классам:
            наследование или куча Util-классов
        \item Проблемы при работе null
        \item Изменяемые структуры данных по-умолчанию
        \item Многословие
    \end{itemize}
\end{frame}

\begin{frame}
    \frametitle{Почему именно Kotlin}
    \begin{itemize}
        \item 100\% совместим с Java
        \item Отличные инструменты разработки
        \item Сравнительно небольшой размер рантайма (\approx 950K) и количество методов (\approx 7900)
    \end{itemize}
\end{frame}

\begin{frame}[fragile, t]
    \frametitle{Базовый синтакс}
    \onslide<1->
    Функции
    \begin{minted}[gobble=8]{kotlin}
        fun sum(a: Int, b: Int): Int {
            return a + b
        }
    \end{minted}

    \onslide<2->
    \begin{minted}[gobble=8]{kotlin}
        fun diff(a: Int, b: Int) = a - b
    \end{minted}

    \onslide<3->
    Переменные
    \begin{minted}[gobble=8]{kotlin}
        val a: Int = 1
        val b = 1 // Автовыведение типа

        var x = 5 // Изменяемая переменная
        x += 1
    \end{minted}
\end{frame}

\begin{frame}[fragile, t]
    \frametitle{Классы, свойства}
    \begin{overprint}
        \onslide<1>
        \begin{minted}[gobble=12]{kotlin}
            class Person {
                val name = "Vasya Pupkin"
            }
        \end{minted}

        \onslide<2>
        \begin{minted}[gobble=12]{kotlin}
            class Person constructor(name: String) {
                val name = name
            }
        \end{minted}

        \onslide<3>
        \begin{minted}[gobble=12]{kotlin}
            class Person constructor(val name: String) {
            }
        \end{minted}

        \onslide<4>
        \begin{minted}[gobble=12]{kotlin}
            class Person(val name: String) {
            }
            val person = Person("Joe Smith")
        \end{minted}
    \end{overprint}
\end{frame}

\begin{frame}[fragile, t]
    \frametitle{Работа с null}
    \onslide<1->
    \begin{minted}[gobble=8]{kotlin}
        var a: String = "abc"
        a = null // ошибка компиляции
    \end{minted}
    \onslide<2->
    \begin{minted}[gobble=8]{kotlin}
        var b: String? = "abc"
        b = null // ok
    \end{minted}
    \onslide<3->
    \begin{minted}[gobble=8]{kotlin}
        val l1 = a.length // ok
        val l2 = b.length // ошибка: 'b' может быть null
    \end{minted}
\end{frame}

\begin{frame}[fragile, t]
    \frametitle{Работа с null}
    \onslide<1->
    Явная проверка на null
    \begin{minted}[gobble=8]{kotlin}
        val l = if (b != null) b.length else -1
    \end{minted}

    \onslide<2->
    Использование оператора ?
    \begin{minted}[gobble=8]{kotlin}
        val l = b?.length
        val name = bob?.department?.head?.name
    \end{minted}

    \onslide<3->
    Использование оператора !!
    \begin{minted}[gobble=8]{kotlin}
        val l = b!!.length()
    \end{minted}
\end{frame}

\begin{frame}[fragile, t]
    \frametitle{Extension functions}
    \onslide<1->
    \begin{minted}[gobble=8]{java}
        static boolean isVeryLong(String s) {
            return s.length() > 80;
        }
        boolean tooLong = TextUtils.isTooLong(s);
    \end{minted}

    \onslide<2->
    \begin{minted}[gobble=8]{kotlin}
        fun String.isTooLong() {
            return length() > 80
        }
        val tooLong = s.isTooLong()
    \end{minted}
\end{frame}

\begin{frame}[fragile, t]
    \frametitle{Лямбды}
    \onslide<1->
    \begin{minted}[gobble=8, fontsize=\small]{java}
        button.setOnClickListener(new View.OnClickListener() {
            public void onCLick(View v) {
                Toast.makeText(MyActivity.this, "Some text",
                Toast.LENGTH_SHORT).show();
            }
        });
    \end{minted}

    \onslide<2->
    \begin{minted}[gobble=8, fontsize=\small]{java}
        button.setOnClickListener(View v ->
            Toast.makeText(MyActivity.this, "Some text",
            Toast.LENGTH_SHORT).show();
        });
    \end{minted}

    \onslide<3->
    \begin{minted}[gobble=8, fontsize=\small]{kotlin}
        button.setOnClickListener({ v: View ->
            Toast.makeText(MyActivity.this, "Some text",
            Toast.LENGTH_SHORT).show()
        })
    \end{minted}

    \onslide<4->
    \begin{minted}[gobble=8, fontsize=\small]{kotlin}
        button.setOnClickListener {
            Toast.makeText(MyActivity.this, "Some text",
            Toast.LENGTH_SHORT).show()
        }
    \end{minted}
\end{frame}


\begin{frame}[fragile, t]
    \frametitle{Функции высшего порядка}
    \begin{overprint}
        \onslide<1>
        \begin{minted}[gobble=12]{kotlin}
            fun <T> List<T>.filter(predicate: (T) -> Boolean): List<T> {
                // ...
            }
        \end{minted}

        \onslide<2>
        \begin{minted}[gobble=12]{kotlin}
            fun <T> List<T>.filter(predicate: (T) -> Boolean): List<T> {
                val resultList = ArrayList<T>()
                for (item in this) {
                    if (predicate(item)) {
                        resultList.add(item)
                    }
                }

                return resultList
            }
        \end{minted}

        \onslide<3>
        \begin{minted}[gobble=12]{kotlin}
            fun <T> List<T>.filter(predicate: (T) -> Boolean): List<T> {
                val resultList = ArrayList<T>()
                for (item in this) {
                    if (predicate(item)) {
                        resultList.add(item)
                    }
                }

                return resultList
            }

            val numbers = listOf(1, 2, 3, 4, 5, 6, 7, 8, 9, 0)
            val oddNumbers = numbers.filter { it % 2 != 0 }
        \end{minted}
    \end{overprint}
\end{frame}

\begin{frame}[fragile, t]
    \frametitle{Функции высшего порядка}
    \begin{overprint}
        \onslide<1>
        \begin{minted}[gobble=12, fontsize=\small]{java}
            db.beginTransaction();
            try {
                db.delete("users", "first_name = ?", new String[] { "Vasya" });
            } finally {
                db.endTransaction();
            }
        \end{minted}

        \onslide<2>
        \begin{minted}[gobble=12, fontsize=\small]{java}
            db.beginTransaction();
            try {
                db.delete("users", "first_name = ?", new String[] { "Vasya" });
                db.setTransactionSuccessful();
            } finally {
                db.endTransaction();
            }
        \end{minted}

        \onslide<3>
        \begin{minted}[gobble=12, fontsize=\small]{kotlin}
            fun SQLiteDatabase.inTransaction(func: () -> Unit) {
                beginTransaction()
                try {
                    func()
                    setTransactionSuccessful()
                } finally {
                    endTransaction()
                }
            }

            db.inTransaction {
                db.delete("users", "first_name = ?", arrayOf("Vasya")
            }
        \end{minted}

        \onslide<4>
        \begin{minted}[gobble=12, fontsize=\small]{kotlin}
            fun SQLiteDatabase.inTransaction(
                        func: (SQLiteDatabase) -> Unit) {
                beginTransaction()
                try {
                    func(this)
                    setTransactionSuccessful()
                } finally {
                    endTransaction()
                }
            }

            db.inTransaction {
                it.delete("users", "first_name = ?", arrayOf("Vasya")
            }
        \end{minted}

        \onslide<5>
        \begin{minted}[gobble=12, fontsize=\small]{kotlin}
            fun SQLiteDatabase.inTransaction(
                    func: SQLiteDatabase.() -> Unit) {
                beginTransaction()
                try {
                    this.func()
                    setTransactionSuccessful()
                } finally {
                    endTransaction()
                }
            }

            db.inTransaction {
                delete("users", "first_name = ?", arrayOf("Vasya")
            }
        \end{minted}

        \onslide<6>
        \begin{minted}[gobble=12, fontsize=\small]{kotlin}
            inline fun SQLiteDatabase.inTransaction(
                    func: SQLiteDatabase.() -> Unit) {
                beginTransaction()
                try {
                    this.func()
                    setTransactionSuccessful()
                } finally {
                    endTransaction()
                }
            }

            db.inTransaction {
                delete("users", "first_name = ?", arrayOf("Vasya")
            }
        \end{minted}
    \end{overprint}
\end{frame}

\begin{frame}
    \frametitle{Документация и полезные ссылки}
    \begin{description}
        \item Сайт и документация: \\
            \url{https://kotlinlang.org}
        \item Онлайн IDE: \\
            \url{http://try.kotlinlang.org}
        \item Презентация Джейка Уортона (Jake Wharton) о языке (англ.): \\
            \href{http://www.youtube.com/watch?v=A2LukgT2mKc}{Android Development with Kotlin}
        \item Мотивации для использования и сравнение с другими языками под JVM (англ.): \\
            \href{https://docs.google.com/document/d/1ReS3ep-hjxWA8kZi0YqDbEhCqTt29hG8P44aA9W0DM8/edit?hl=en&forcehl=1}{Using Project Kotlin for Android}
    \end{description}
\end{frame}

\end{document}
